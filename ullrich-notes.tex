% Document setup
\documentclass[article, a4paper, 11pt, oneside]{memoir}
\usepackage[utf8]{inputenc}
\usepackage[T1]{fontenc}
\usepackage[UKenglish]{babel}

% Document info
\newcommand\doctitle{Ullrich, \emph{Complex Made Simple}}
\newcommand\docauthor{Danny Nygård Hansen}

% Formatting and layout
\usepackage[autostyle]{csquotes}
\usepackage[final]{microtype}
\usepackage{xcolor}
\frenchspacing
\usepackage{latex-sty/articlepagestyle}
\usepackage{latex-sty/articlesectionstyle}
% \usepackage{latex-sty/amalgsymbol}

% Fonts
\usepackage[largesmallcaps]{kpfonts}
\DeclareSymbolFontAlphabet{\mathrm}{operators} % https://tex.stackexchange.com/questions/40874/kpfonts-siunitx-and-math-alphabets
\linespread{1.06}
\let\mathfrak\undefined
\usepackage{eufrak}
\usepackage{inconsolata}
% \usepackage{amssymb}

% Hyperlinks
\usepackage{hyperref}
\definecolor{linkcolor}{HTML}{4f4fa3}
\hypersetup{%
	pdftitle=\doctitle,
	pdfauthor=\docauthor,
	colorlinks,
	linkcolor=linkcolor,
	citecolor=linkcolor,
	urlcolor=linkcolor,
	bookmarksnumbered=true
}

% Equation numbering
\numberwithin{equation}{chapter}

% Footnotes
\footmarkstyle{\textsuperscript{#1}\hspace{0.25em}}

% Mathematics
\usepackage{latex-sty/basicmathcommands}
\usepackage{latex-sty/framedtheorems}
\usepackage{latex-sty/theoremrefs}
\usepackage{latex-sty/topologycommands}
\usepackage{tikz-cd}
\tikzcdset{arrow style=math font} % https://tex.stackexchange.com/questions/300352/equalities-look-broken-with-tikz-cd-and-math-font
\usetikzlibrary{babel}

% Lists
\usepackage{enumitem}
\setenumerate[0]{label=\normalfont(\roman*)}
\setlist{  
  listparindent=\parindent,
  parsep=0pt,
}

% Bibliography
\usepackage[backend=biber, style=authoryear, maxcitenames=2, useprefix]{biblatex}
\addbibresource{references.bib}

% Title
\title{\doctitle}
\author{\docauthor}

\newcommand{\setF}{\mathbb{F}}
\newcommand{\ev}{\mathrm{ev}}
\newcommand{\calT}{\mathcal{T}}
\newcommand{\calU}{\mathcal{U}}
\newcommand{\calB}{\mathcal{B}}
\newcommand{\calE}{\mathcal{E}}
\newcommand{\calC}{\mathcal{C}}
\newcommand{\calD}{\mathcal{D}}
\newcommand{\calF}{\mathcal{F}}
\newcommand{\calG}{\mathcal{G}}
\newcommand{\calM}{\mathcal{M}}
\newcommand{\calA}{\mathcal{A}}
\newcommand{\calP}{\mathcal{P}}
\newcommand{\calR}{\mathcal{R}}
\newcommand{\calO}{\mathcal{O}}
\newcommand{\strucS}{\mathfrak{S}}
\DeclarePairedDelimiter{\gen}{\langle}{\rangle} % Generating set
\newcommand{\frakL}{\mathfrak{L}}
\newcommand{\frakN}{\mathfrak{N}}
\newcommand{\frakA}{\mathfrak{A}}
\newcommand{\frakB}{\mathfrak{B}}
\newcommand{\ab}{\mathit{ab}}

\DeclareMathOperator{\im}{im}
\DeclareMathOperator{\coker}{coker}
\DeclareMathOperator{\stab}{Stab}

% Categories
\newcommand{\cat}[1]{\mathcal{#1}}
\newcommand{\scat}[1]{\mathbf{#1}} % category supposed to be small
\newcommand{\ncat}[1]{\mathbf{#1}} % named categories like Set, Top

\newcommand{\catSet}{\ncat{Set}} % Category of sets
\newcommand{\catGrp}{\ncat{Grp}} % Category of groups
\newcommand{\catAb}{\ncat{Ab}} % Category of abelian groups
\newcommand{\catRing}{\ncat{Ring}} % Category of rings
\newcommand{\catFld}{\ncat{Fld}} % Category of fields

\newcommand{\catMod}[1]{{#1\text{-}\scat{Mod}}}
\newcommand{\catRMod}{\catMod{R}}

\newcommand{\End}{\mathrm{End}}
\newcommand{\Hom}{\mathrm{Hom}}

\DeclareMathOperator{\chr}{char}


%% Framed exercise environment

\mdfdefinestyle{swannexercise}{%
    skipabove=0.5em plus 0.4em minus 0.2em,
	skipbelow=0.5em plus 0.4em minus 0.2em,
	leftmargin=-5pt,
	rightmargin=-5pt,
	innerleftmargin=5pt,
	innerrightmargin=5pt,
	innertopmargin=5pt,
	innerbottommargin=4pt,
	linewidth=0pt,
	splittopskip=1.2em minus 0.2em,
	splitbottomskip=0.5em plus 0.2em minus 0.1em,
	backgroundcolor=backgroundcolor,
	frametitlebackgroundcolor=titlecolor,
	frametitlefont={\scshape},
    theoremseparator={},
    % theoremspace={},
	frametitleaboveskip=3pt,
	frametitlebelowskip=2pt
}

\mdtheorem[style=swannexercise]{exerciseframed}{Exercise}

\let\oldexerciseframed\exerciseframed
\renewcommand{\exerciseframed}{%
  \crefalias{theorem}{exerciseframed}%
  \oldexerciseframed}

\usepackage{listofitems}

\settocdepth{subsection}
\renewenvironment{exerciseframed}[1][]{%
    \setsepchar{.}%
    \readlist*\mylist{#1}%
    \def\smalllabel{\mylist[2].\mylist[3]}%
    \refstepcounter{exerciseframed}%
    % \addcontentsline{toc}{subsection}{Exercise \smalllabel}%
    \begin{exerciseframed*}[#1]%
    \label{ex:#1}%
}{%
    \end{exerciseframed*}%
}

% https://tex.stackexchange.com/a/23491/63353
\newcommand{\RNum}[1]{\uppercase\expandafter{\romannumeral #1\relax}}

\newcommand{\exref}[1]{%
    % \setsepchar{.}%
    % \readlist*\mylist{#1}%
    % \ifnum \arabic{chapter}=\mylist[1]
    %     \def\mylabel{\mylist[2].\mylist[3]}%
    % \else
    %     \def\mylabel{\RNum{\mylist[1]}.\mylist[2].\mylist[3]}%
    % \fi
    \hyperref[ex:#1]{Exercise~#1}%
}

\theoremstyle{nonumberplain}
\theoremsymbol{\ensuremath{\square}}
\newtheorem{solution}{Solution}

\let\oldsolution\solution
\renewcommand{\solution}{%
  \crefalias{theorem}{solution}%
  \oldsolution}

\newcommand{\solutionlabelfont}[1]{{\normalfont\color{linkcolor}#1}}
\newlist{solutionsec}{enumerate}{1}
\setlist[solutionsec]{leftmargin=0pt, parsep=0pt, listparindent=\parindent, font=\solutionlabelfont, label=(\roman*), labelsep=0pt, labelwidth=20pt, itemindent=20pt, align=left, itemsep=10pt}


% \renewcommand{\thechapter}{\Roman{chapter}}
% \renewcommand{\thesection}{\arabic{section}}

\DeclarePairedDelimiter{\ord}{\lvert}{\rvert}
\DeclareMathOperator{\lcm}{lcm}
\DeclareMathOperator{\Aut}{Aut}
\DeclareMathOperator{\Inn}{Inn}

\usepackage{caption} % Links to figures jump correctly
\Crefname{figure}{Figure}{Figures}


\newenvironment{displaytheorem}{%
	\begin{displayquote}\itshape%
}{%
	\end{displayquote}%
}


\newcommand{\upset}{\operatorname{\uparrow}}
\newcommand{\downset}{\operatorname{\downarrow}}
\newcommand{\matgroup}[3]{\mathrm{#1}_{#2}(#3)}
\newcommand{\GL}[2]{\matgroup{GL}{#1}{#2}}
\newcommand{\SL}[2]{\matgroup{SL}{#1}{#2}}
\newcommand{\PSL}[2]{\matgroup{PSL}{#1}{#2}}
\newcommand{\catGSet}[1][G]{{#1\text{-}\catSet}}
\newcommand{\frakI}{\mathfrak{I}}
\newcommand{\field}{\mathbb{F}}
\let\bigcoprod\coprod
\renewcommand{\coprod}{\sqcup}


% TODO for logic
\renewcommand{\mylistlabelfont}[1]{{\normalfont\color{linkcolor}\textit{#1}:}}
\newlist{notelist}{description}{1}
\setlist[notelist]{leftmargin=0pt, parsep=0pt, listparindent=\parindent, font=\mylistlabelfont}


\begin{document}

\maketitle

\setcounter{chapter}{-1}
\chapter{Differentiability and the Cauchy--Riemann Equations}

\begin{exerciseframed}[0.3]
    Suppose that $T \colon \complex \to \complex$ is $\reals$-linear. Show that $T$ is $\complex$-linear if and only if $T(\iu z) = \iu Tz$ for all $z$.
\end{exerciseframed}

\begin{solution}
    The \enquote{if} part is obvious, so we prove the \enquote{only if} part. If $T(\iu z) = \iu Tz$ for all $z$, then for $a,b \in \reals$ we have
    %
    \begin{equation*}
        T((a + \iu b)z)
            = T(az + \iu bz)
            = a Tz + \iu T(bz)
            = a Tz + \iu b Tz
            = (a + \iu b) Tz,
    \end{equation*}
    %
    as desired.
\end{solution}


\begin{exerciseframed}[0.4]
    Suppose that $T \colon \reals^2 \to \reals^2$ is the $\reals$-linear mapping defined by the matrix
    %
    \begin{equation*}
        \begin{pmatrix}
            a & b \\
            c & d
        \end{pmatrix}.
    \end{equation*}
    %
    Show that $T$ is $\complex$-linear if and only if $a = d$ and $b = -c$.
\end{exerciseframed}

\begin{solution}
    Notice that, for $a,b,x,y \in \reals$,
    %
    \begin{equation*}
        \begin{pmatrix}
            a \\
            b
        \end{pmatrix}
        \cdot_\complex
        \begin{pmatrix}
            x \\
            y
        \end{pmatrix}
        =
        \begin{pmatrix}
            ax - by \\
            ay + bx
        \end{pmatrix}
        =
        \begin{pmatrix}
            a & -b \\
            b & a
        \end{pmatrix}
        \begin{pmatrix}
            x \\
            y
        \end{pmatrix}.
    \end{equation*}
    %
    That is, multiplication by a complex number $a + \iu b$ is the same as ordinary matrix multiplication by the matrix
    %
    \begin{equation*}
        \begin{pmatrix}
            a & -b \\
            b & a
        \end{pmatrix}.
    \end{equation*}
    %
    Since matrix multiplication is associative, by [TODO ref] the requirement is that $T$ commutes with the matrix representing the imaginary unit $\iu$. This means that
    %
    \begin{equation*}
        \begin{pmatrix}
            a & b \\
            c & d
        \end{pmatrix}
        \begin{pmatrix}
            0 & -1 \\
            1 & 0
        \end{pmatrix}
        =
        \begin{pmatrix}
            b & -a \\
            d & -c
        \end{pmatrix}
        \quad \text{should equal} \quad
        \begin{pmatrix}
            0 & -1 \\
            1 & 0
        \end{pmatrix}
        \begin{pmatrix}
            a & b \\
            c & d
        \end{pmatrix}
        =
        \begin{pmatrix}
            -c & -d \\
            a & b
        \end{pmatrix}.
    \end{equation*}
    %
    And this is the case if and only if $a = d$ and $b = -c$ as claimed.

    Note: The subring of $\mathrm{Mat}_2(\reals)$ representing $\complex$-linear maps is precisely the centraliser of the matrix of $\iu$.
\end{solution}


\chapter{Power Series}

\chapter{Preliminary Results on Holomorphic Functions}

\chapter{Elementary Results on Holomorphic Functions}

\begin{remarkbreak}[Topological lemmas]
    We give alternative proofs of Lemmas~3.3 and 3.4, using the covering definition of compactness.

    For Lemma~3.3, let $d$ denote the Euclidean metric on $\complex$, and write $d(z,A)$ for the distance from $z \in \complex$ to $A \subseteq \complex$. It is well-known that this is continuous in $z$ for a fixed $A$, and that $d(z,A) = 0$ if and only if $z \in \closure{A}$. Since $K$ is compact, this obtains its minimum at some $z_0 \in K$. But then we must have $d(z_0,V^c) > 0$ since $V^c$ is closed. Then choose $\rho = d(z_0,V^c)/2$.

    For Lemma~3.4 we use the Heine--Borel theorem. Since $K'$ is obviously bounded, it suffices to show that it is closed. To this end, let $w \in \complex \setminus K'$. The map $z \mapsto d(w,z)$ obtains its minimum $d(w,K)$ on $K$ at a point $z_0$. Notice that $d(w,K) > \rho$, since otherwise we would have $w \in \closure{D}(z_0,\rho) \subseteq K'$. Hence $D(w,d(w,K)-\rho)$ is disjoint from $K'$, so $\complex \setminus K'$ is open.
\end{remarkbreak}


\begin{remarkbreak}[Order of zeros]
    \label{rem:order-of-zeros}
    We show that the implication in Lemma~3.9 is an equivalence, i.e. that
    %
    \begin{displaytheorem}
        A function $f \in H(V)$ has a zero of order $N$ at $z \in V$ if and only if there exists a $g \in H(V)$ with $g(z) \neq 0$ such that
        %
        \begin{equation*}
            f(w)
                = (w - z)^N g(w)
        \end{equation*}
        %
        for all $w \in V$.
    \end{displaytheorem}
    %
    The \textquote{only if} direction is just Lemma~3.9, so we prove the converse. In this case, the Leibniz differentiation rule implies that
    %
    \begin{equation*}
        f^{(n)}(w)
            = \sum_{k=0}^n \binom{n}{k} g^{(n-k)}(w) \frac{\dif^{k}}{\dif w^k} (w - z)^N
    \end{equation*}
    %
    for all $n \in \naturals_0$. If $n < N$, then the $n$th derivative of $(w-z)^N$ is zero at $z$, so $f^{(n)}(z) = 0$. But if $n = N$, then all terms except the term $k = n$ vanish, so
    %
    \begin{equation*}
        f^{(N)}(w)
            = g(w) \frac{\dif^{N}}{\dif w^N} (w - z)^N
            = g(w) N!,
    \end{equation*}
    %
    so $f^{(N)}(z) \neq 0$. Hence $f$ has a zero of order $N$ at $z$.
\end{remarkbreak}


\begin{exerciseframed}[3.9]
    Suppose that $V$ is a connected open set, $f \in H(V)$, and $f'(z) = 0$ for all $z \in V$. Show that $f$ is constant.
\end{exerciseframed}

\begin{solution}
    Let $D(z_0,r) \subseteq V$. For $a,b \in D(z_0,r)$ we have
    %
    \begin{equation*}
        f(b) - f(a)
            = \int_{[a,b]} f'(z) \dif z
            = 0,
    \end{equation*}
    %
    so $f$ is constant on $D(z_0,r)$. Hence all derivatives of $f$ vanish at $z_0$, so Corollary~3.8 implies that $f$ is constant.
\end{solution}


\begin{exerciseframed}[3.19]
    Suppose that $f \in H(D'(z_0,r))$. Show that $f$ has a pole of order $N$ at $z_0$ if and only if $1/f$ has a zero of order $N$ at $z_0$.
\end{exerciseframed}

\begin{solution}
    First assume that $f$ has a pole of order $N$ at $z_0$. Then $f$ is on the form
    %
    \begin{equation*}
        f(z)
            = \sum_{n=-N}^\infty c_n (z-z_0)^n
            = (z-z_0)^{-N} \sum_{n=0}^\infty c_{n-N} (z-z_0)^n
            = (z-z_0)^{-N} h(z)
    \end{equation*}
    %
    for $z \in D'(z_0,r)$, where $h(z) = \sum_{n=0}^\infty c_{n-N} (z-z_0)^n$ is holomorphic in $D(z_0,r)$ (it is holomorphic in the punctured disk, but it is a power series with a positive radius of convergence, hence is differentiable at $0$). Notice that $h(z_0) = c_{-N} \neq 0$. But then
    %
    \begin{equation*}
        \frac{1}{f(z)}
            = (z-z_0)^N \frac{1}{h(z)}
    \end{equation*}
    %
    in a neighbourhood of $z_0$ (since $1/f$ has a removable singularity at $z_0$ by Lemma~3.13). And since $h(z_0) \neq 0$, $1/h$ is holomorphic in a neighbourhood of $z_0$. The claim now follows from \cref{rem:order-of-zeros}.

    Now assume that $1/f$ has a zero of order $N$ at $z_0$. Then there is a $g \in H(D'(z_0,r))$ with $g(z_0) \neq 0$ such that
    %
    \begin{equation*}
        \frac{1}{f(w)}
            = (z - w)^N g(w),
    \end{equation*}
    %
    by Lemma~3.9. But then
    %
    \begin{equation*}
        f(w)
            = (z - w)^{-N} \frac{1}{g(w)},
    \end{equation*}
    %
    and $1/g$ is holomorphic near $z_0$ since it is nonzero there. Hence it is given by a power series, and the claim follows.
\end{solution}


\begin{exerciseframed}[3.20]
    Suppose we agree that a pole of order $N$ is a zero of order $-N$, and that if $f$ is holomorphic near $z_0$ and $f(z_0) \neq 0$ then $f$ has a zero of order $0$ at $z_0$. Suppose that $f$ and $g$ are holomorphic in $D'(z_0, r)$ and have zeroes of order $n, m \in \naturals$ at $z_0$, respectively. Show that $fg$ has a zero of order $n + m$ at $z_0$.
\end{exerciseframed}

\begin{solution}
    If $n,m \geq 0$, then the claim follows easily from \cref{rem:order-of-zeros}. By [TODO ref] Exercise~3.19 we only need to consider the case where $n > 0$ and $m \leq 0$. We thus have
    %
    \begin{equation*}
        f(z)
            = (z-z_0)^n h(z)
        \quad \text{and} \quad
        g(z)
            = \sum_{k=m}^\infty c_k (z-z_0)^k,
    \end{equation*}
    %
    where $h$ is holomorphic near $z_0$ and $h(z_0) \neq 0$, and $c_m \neq 0$. But then
    %
    \begin{equation*}
        f(z)g(z)
            = (z-z_0)^{n+m} h(z) \sum_{k=0}^\infty c_{k+m} (z-z_0)^k,
    \end{equation*}
    %
    and since $h(z) \sum_{k=0}^\infty c_{k+m} (z-z_0)^k$ is holomorphic and nonzero at $z = z_0$ (since $h(z_0)$ and $c_m$ are nonzero), it follows from \cref{rem:order-of-zeros} that $fg$ has a zero of order $n+m > 0$ at $z_0$.
\end{solution}


\chapter{Logarithms, Winding Numbers and Cauchy's Theorem}

\newcommand{\Ind}{\mathrm{Ind}}
\newcommand{\Res}{\mathrm{Res}}

\begin{exerciseframed}[4.1]
    Show directly from the definition that $\Ind(\boundary D(a,r), a) = 1$ for any $a \in \complex$ and $r > 0$.
\end{exerciseframed}

\begin{solution}
    Since $\boundary D(a,r)$ is smooth we have
    %
    \begin{equation*}
        \Ind(\boundary D(a,r), a)
            = \frac{1}{2\pi\iu} \int_{\boundary D(a,r)} \frac{\dif z}{z - a}
            = \frac{1}{2\pi\iu} \int_0^1 \frac{2\pi\iu\e^{2\pi\iu t}}{\e^{2\pi\iu t}} \dif t
            = 1.
    \end{equation*}
\end{solution}


\begin{exerciseframed}[4.2]
    Suppose $a \in \complex$ and $r > 0$. Show that
    %
    \begin{equation*}
        \Ind(\boundary D(a,r), z) =
        \begin{cases}
            1, & \abs{z - a} < r, \\
            0, & \abs{z - a} > r.
        \end{cases}
    \end{equation*}
\end{exerciseframed}

\begin{solution}
    By Proposition~4.7, the index is constant on the components of $\complex \setminus \boundary D(a,r)$ and $0$ on the unbounded component. The claim then follows from Exercise~4.1.
\end{solution}


\begin{exerciseframed}[4.7]
    \begin{enumerate}
        \item Suppose that $f$ and $g$ are holomorphic near $z_0$ and $f$ has a simple zero at $z_0$. Find an expression for the residue of $g/f$ at $z_0$.
        
        \item Suppose that $f$ has a simple pole at $z_0$ and $g$ is holomorphic near $z_0$. Show that
        %
        \begin{equation*}
            \Res(fg, z_0)
                = g(z_0) \Res(f, z_0).
        \end{equation*}
        
        \item Suppose that $f$ is holomorphic in a neighborhood of $z_0$, set $g(z) = f(z)/(z-z_0)^n$, and show that
        %
        \begin{equation*}
            \Res(g,z_0)
                = f^{(n-1)}(z_0)/(n-1)!.
        \end{equation*}
    \end{enumerate}
\end{exerciseframed}

\begin{solution}
\begin{solutionsec}
    \item By [TODO ref] Exercise~3.20 $g/f$ has a zero of order $N \geq -1$ at $z_0$. If $N = -1$, the residue is then, by the calculations on p.~71,
    %
    \begin{equation*}
        \Res(g/f, z_0)
            = \lim_{z \to z_0} (z - z_0) \frac{g(z)}{f(z)}
            = \lim_{z \to z_0} g(z) \frac{z - z_0}{f(z) - f(z_0)}
            = \frac{g(z_0)}{f'(z_0)}.
    \end{equation*}
    %
    If instead $N \geq 0$, then $g$ has a zero of order at least $1$ at $z_0$, in which case $g(z_0)$. But in this case $g/f$ has a removable singularity at $z_0$, so its residue at $z_0$ is $0$, so the formula also holds in this case.

    \item Notice that $1/f$ has a simple zero at $z_0$ by Exercise~3.20, so part (i) implies that
    %
    \begin{align*}
        \Res(fg, z_0)
            &= \frac{g(z_0)}{(1/f)'(z_0)} \\
            &= g(z_0) \frac{1}{(1/f)'(z_0)} \\
            &= g(z_0) \Res(1/(1/f), z_0) \\
            &= g(z_0) \Res(f, z_0).
    \end{align*}

    \item This is obvious from the calculations on p.~71.
\end{solutionsec}
\end{solution}


\begin{exerciseframed}[4.8]
    Suppose that $f$ has an isolated singularity at $z_0$. Fix $r > 0$ and $n \in \ints$, and define $\gamma \colon [0, 2\pi] \to \complex$ by
    %
    \begin{equation*}
        \gamma(t)
            = z_0 + r \e^{\iu n t}.
    \end{equation*}
    %
    Show that $\Ind(\gamma,z_0) = n$ and that
    %
    \begin{equation*}
        \frac{1}{2\pi\iu} \int_\gamma f(z) \dif z
            = \Ind(\gamma,z_0) \Res(f,z_0)
    \end{equation*}
    %
    if $r > 0$ is small enough.
\end{exerciseframed}

\begin{solution}
    Both sides are zero if $n = 0$, so we may assume that $n \neq 0$. Since $\gamma$ is smooth we have
    %
    \begin{equation*}
        \Ind(\gamma, z_0)
            = \frac{1}{2\pi\iu} \int_{\gamma} \frac{\dif z}{z - a}
            = \frac{1}{2\pi\iu} \int_0^{2\pi} \frac{\iu n\e^{\iu nt}}{\e^{\iu nt}} \dif t
            = n.
    \end{equation*}
    %
    Assume that $n > 0$ and notice that $\gamma$ is an $n$-fold sum
    %
    \begin{equation*}
        \gamma
            = \boundary D(z_0,r) \mathbin{\dot{+}} \cdots \mathbin{\dot{+}} \boundary D(z_0,r),
    \end{equation*}
    %
    so that
    %
    \begin{align*}
        \frac{1}{2\pi\iu} \int_\gamma f(z) \dif z
            &= \frac{n}{2\pi\iu} \int_{\boundary D(z_0,r)} f(z) \dif z \\
            &= n \Res(f,z_0) \\
            &= \Ind(\gamma,z_0) \Res(f,z_0),
    \end{align*}
    %
    where the second equality follows from Theorem~4.11 as on p.~72. Here we choose $r$ such that $r < R$ and $f$ is holomorphic on $D'(z_0,R)$ for some $R$. Substituting $-n$ for $n$ changes sign on the integral on the left-hand side and on $\Ind(\gamma,z_0)$, proving the claim.
\end{solution}


\addtocounter{chapter}{3}
\chapter{Conformal Mappings}

\newcommand{\disk}{\mathbb{D}}

\begin{exerciseframed}[8.7]
    Suppose that $\psi \in \Aut(\Pi^+)$. Show that there exist $a,b,c,d \in \reals$ with $ad-bc = 1$, such that
    %
    \begin{equation*}
        \psi(z)
            = \frac{az + b}{cz + d}
    \end{equation*}
    %
    for all $z \in \Pi^+$. Show that $a,b,c,d$ are not unique but are almost unique.
\end{exerciseframed}

\begin{solution}
    First notice that $\phi_C\inv \circ \psi \circ \phi_C \in \Aut(\disk)$, where $\phi_C \colon \disk \to \Pi^+$ is the Cayley transform. Hence this is a Möbius transformation by Theorem~8.4.3, so $\psi$ is as well.

    Since $\psi$ is also a homeomorphism $\complex_\infty \to \complex_\infty$, it preserves the boundary of the upper hemisphere, which is precisely the one-point compactification $\reals^*$ of the real line. For $i = 1,2,3$ choose distinct $z_i \in \reals \setminus \{d/c\}$ such that $w_i \defn \psi(z_i) \in \reals$. Then the $w_i$ are also distinct, so Theorem~8.2.2 implies that $\psi$ is unique among Möbius transformations with this property. Now recall that the coefficients $a,b,c,d$ are unique up to multiplication by a nonzero scalar, so we may assume that $b \in \{0,1\}$. Given $b$ the remaining coefficients are thus uniquely determined. We claim that they are also real.

    Consider the linear system of equations
    %
    \begin{equation*}
        \begin{pmatrix}
            -z_1 & z_1 w_1 & w_1 \\
            -z_2 & z_2 w_2 & w_2 \\
            -z_3 & z_3 w_3 & w_3
        \end{pmatrix}
        \begin{pmatrix}
            a \\ c \\ d
        \end{pmatrix}
        =
        \begin{pmatrix}
            b \\ b \\ b
        \end{pmatrix}
    \end{equation*}
    %
    in the variables $a,c,d$. This has a unique solution by uniqueness of $\psi$, so the matrix on the left-hand side is invertible. But it has real entries, and so does the vector on the right-hand side, so $a,c,d$ must also be real.

    Next notice that if $z \in \Pi^+$ then
    %
    \begin{equation*}
        \frac{az + b}{cz + d}
            = \frac{ac \abs{z}^2 + bd + adz + bc\conj{z}}{\abs{cz + d}^2},
    \end{equation*}
    %
    so
    %
    \begin{equation*}
        \Im \frac{az + b}{cz + d}   
            = \frac{(ad-bc) \Im z}{\abs{cz + d}^2},
    \end{equation*}
    %
    and for this to lie in $\Pi^+$ we must have $ad-bc > 0$. Hence by dividing each coefficient by $\sqrt{ad - bc}$ we may assume that $ad - bc = 1$.
    
    Furthermore, this argument shows that any Möbius transformation $\phi$ with real coefficients and determinant $1$ maps $\Pi^+$ into itself. But its inverse as a map $\complex_\infty \to \complex_\infty$ is also a Möbius transformation with real coefficients and determinant $1$, so this is also an endomorphism on $\Pi^+$. Hence $\phi$ is in fact an automorphism on $\Pi^+$.

    Finally, consider the (well-defined) homomorphism $\SL{2}{\reals} \to \Aut(\Pi^+)$ given by $A \mapsto \phi_A$. Its kernel is the subgroup of real scalar multiples of the identity matrix $I$ whose determinant is $1$. But this is just $\{I,-I\}$, so we find that $\Aut(\Pi^+) \cong \SL{2}{\reals}/\{I,-I\} = \PSL{2}{\reals}$.
\end{solution}

\end{document}